%-------------------------------------------------------------------------------
%	SECTION TITLE
%-------------------------------------------------------------------------------
\cvsection{Experience}


%-------------------------------------------------------------------------------
%	CONTENT
%-------------------------------------------------------------------------------
\begin{cventries}

%---------------------------------------------------------
  \cventry
    {Postdoctoral Researcher} % Job title
    {CNRS -- Université de Nantes} % Organization
    {Nantes, France} % Location
    {Jan. 2018 - Dec. 2019} % Date(s)
    {
	Mentor: Dr. Samuel Chaffron (team COMBI, LS2N, CNRS).
    \vspace{0.5cm}
	\begin{cvitems} % Description(s) of tasks/responsibilities
		\item {Constructed a global network of microbial interactions in the global ocean from Tara expeditions metagenomic and metatranscriptomic data.}
		\item {Identified communities of co-active cultivated and non-cultivated microbes (bacteria/archaea).}
		\item {Uncovered functional traits linked to mutualism, including cross-feeding interactions derived from metabolic network reconstruction.} 
	\end{cvitems}
    }
    
%---------------------------------------------------------
  \cventry
    {Research and Teaching Assistant, PhD Student} % Job title
    {Inria -- Université Grenoble Alpes} % Organization
    {Grenoble, France} % Location
    {Sep. 2012 - Mar. 2017} % Date(s)
    {
	Supervisors: Dr. Hidde de Jong (Project-team Ibis, Inria) \& Pr. Johannes Geiselmann (team BIOP, LIPhy).
    \vspace{0.5cm}
	\begin{cvitems} % Description(s) of tasks/responsibilities
        \item {Constructed an abstract mathematical model of nutrient allocation in a microorganism.}
        \item {Applied Optimal Control to predict the optimal regulation of nutrient allocation during an environmental change.}
        \item {Showed that such a regulation is reminiscent of known regulatory processes in E. \textit{coli} (published in \textit{Plos Comp. Biol.}).}
		\item {Engineered bacteria with fluorescent ribosomes and monitored them during an environmental change using a microfluidic device.}
	\end{cvitems}
    }
\end{cventries}    

\cvsubsection{Internships}

\begin{cventries}
%------------------------------------------------
  \cventry
    {Research Assistant, Intern} % Job title
    {Inria} % Organization
    {Grenoble, France} % Location
    {Feb.-Jun. 2012} % Date(s)
    {
    Supervisors: Dr. Hidde de Jong \& Dr. Delphine Ropers (Project-team Ibis, Inria).
    \vspace{0.5cm}
	\begin{cvitems} % Description(s) of tasks/responsibilities
		\item {Reviewed and implemented state-of-the-art methods of sensitivity analysis on a complex model of the gene expression machinery in E. \textit{coli}.}
		\item {Developed a brand-new dynamical method of global sensitivity analysis.}
		\item {Helped to identify the key parameters driving the model dynamics and to reduce its complexity.}
	\end{cvitems}
    }
    
%------------------------------------------------
  \cventry
    {Research Assistant, Intern} % Job title
    {University of Cambridge} % Organization
    {Cambridge, United Kingdom} % Location
    {Feb.-Jun. 2011} % Date(s)
    {
	Supervisor: Pr. Raymond E. Goldstein, Department of Applied Mathematics and Theoretical Physics.
    \vspace{0.5cm}
	\begin{cvitems} % Description(s) of tasks/responsibilities
		\item {Led a theoretical study about the evolution towards multicellularity in microalgae (Volvocales)}
		\item {Developed a general model of phosphate uptake and growth in microalgae}
		\item {Explored the role of the extracellular matrix for phosphate storage, especially in changing environmental conditions.}
	\end{cvitems}
    }
    
%------------------------------------------------
  \cventry
    {Research Assistant, Intern} % Job title
    {École Normale Supérieure} % Organization
    {Paris, France} % Location
    {Jun.-Jul. 2010} % Date(s)
    {
	Supervisor: Dr. Silvia de Monte (Eco-evolutionary Mathematics, IBENS).
    \vspace{0.5cm}
	\begin{cvitems} % Description(s) of tasks/responsibilities
		\item {Analyzed time-series fluorescence data of oscillating yeast cells suspensions subject to periodic forcing.}
		\item {Modified an existing mathematical model based on Hopf bifurcation to recreate the observed dynamics.}
		\item {Showed that even when an irrational forcing is applied, the biological system does not exhibit any chaotic behavior (published in \textit{JCIS}).}
	\end{cvitems}
    }

\end{cventries}

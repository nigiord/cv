\cvsection{Publications}

\begin{refsection}
	\nocite{temporaryGiordanoCoactivity2022}
	\nocite{aliliCharacterizationGutMicrobiota2022}
	\nocite{dejongMathematicalModellingMicrobes2017}
	\nocite{giordanoDynamicalAllocationCellular2016}
	\nocite{giordanoDynamicalResponsesOscillating2012}

	\printbibliography[
	heading=none, 
	]
\end{refsection}
